\documentclass[a4paper,10pt]{extreport}
\usepackage{listings}
\usepackage[utf8]{inputenc}
\usepackage[T1]{fontenc}
\usepackage[francais]{babel}
\usepackage{fancyhdr}
\usepackage{xcolor}



\setlength{\headheight}{14.998pt}
\pagestyle{fancy}
\fancyhead[L]{Brandon Cousin, Tristan Biardeau, Ewen Chaudemanche}
\fancyhead[R]{	Université du maine}

\lstset{
  belowcaptionskip=1\baselineskip,
  breaklines=true,
  tabsize=4,
  language=C,
  showstringspaces=false,
  basicstyle=\footnotesize\ttfamily,
  keywordstyle=\bfseries\color{green!40!black},
  commentstyle=\itshape\color{purple!40!black},
  identifierstyle=\color{blue},
  stringstyle=\color{orange}
}

\begin{document}

\part{Affichage}

\chapter{Menu.c}
\vspace{-1cm}

\begin{lstlisting}[language=c]

void mainMenu(){
	/*
		Afficher menu principal
		1 - Nouvelle Partie
		2 - Charger une Partie
		3 - Quitter
	 */
}

void gameMenu(int listUnitP, movable, attackable){
	/*
		Afficher le menu de jeu
		1 - Unite pouvant se deplacer -> unitMenu();
		2 - Unite pouvant attaquer -> unitMenu();
		3 - Changer de direction -> unitMenu();
		4 - Passer tour
		5 - Abandonner la partie
	 */
}

void unitMenu(int choice, int listUnitP, movable, attackable){
	/*
		Afficher la liste des unites pouvant faire quelque chose
		Si choice-> 1 alors liste des unites pouvant se deplacer
		Si choice -> 2 alors liste des unites pouvant attaquer
		Si choice -> 3 alors liste de toutes les unites du joueur
		Appui sur une touche fait appel a playTurn();
	 */
}
\end{lstlisting}

\chapter{Grid.c}
\vspace{-1cm}

\begin{lstlisting}[language=c]

void gridDisp(){
	// Affiche la grille
}

\end{lstlisting}

\part{Moteur de jeu}

\chapter{gameEngine.c}
\vspace{-1cm}

\begin{lstlisting}[language=c]

int gameInit(int listUnitP, int * noPlayer){
	gridInit(); // Retourne un code d'erreur
	playerInit(listUnitP); // Retourne un code d'erreur
	* noPlayer = rand(1,2);
	return true; // Si les fonctions ne retournent rien
}

vector selectUnit(int unitSelected, int listUnitP){
	// Selectionne l'unite -> coordonnees dans la matrice sous forme de vecteur
}

void playTurn(int unitSelected, int listUnitP, movable){
	selectUnit(int unitSelected, listUnitP); // Renvoie les coordonnees de l'unite selectionnee
	playMove(unitSelected);
	playAttack(unitSelected, listUnitP);
	playDirection(unitSelected, listUnitP);
}

bool endGame(int listUnitP){
	/* 	
		Test les conditions de victoire ou de match nul
	 	Affiche le vainqueur le cas echeant
	*/
}

int movable(unitAction movable[]){
	/* 	
		Retourne nombre d'unites deplacable + liste maj des unites pouvant se deplacer
	*/
}

void timeTurn(){
	// Gestion du temps
}

int lookAround(vector currentUnit){
	/*
		Regarde autour de l'unite selectionnee si elle peut se deplacer sur une case si TP pas permise alors deplacement impossible
	 */
}

void playMove(vector unitSelected, path[]){
	/*
		Deplace l'unite selectionnee a l'endroit desire en prenant en compte les chemins possibles
	 */
}

void playAttack(vector unitSelected, int listUnitP){
	/*
		Attaque avec l'aide de l'unite selectionnee une unite contenu dans la liste listUnitP
	 */
}

void playDirection(vector unitSelected){
	//Change la direction de l'unite selectionnee
}

\end{lstlisting}

\end{document}