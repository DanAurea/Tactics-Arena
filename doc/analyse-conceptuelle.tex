\documentclass[a4paper,10pt]{extreport}
\usepackage{listings}
\usepackage[utf8]{inputenc}
\usepackage[T1]{fontenc}
\usepackage[francais]{babel}
\usepackage{fancyhdr}
\usepackage{xcolor}
\usepackage{pgfgantt}



\setlength{\headheight}{14.998pt}
\pagestyle{fancy}
\fancyhead[L]{Brandon Cousin, Tristan Biardeau, Ewen Chaudemanche}
\fancyhead[R]{	Université du maine}

\lstset{
  belowcaptionskip=1\baselineskip,
  breaklines=true,
  tabsize=4,
  language=C,
  showstringspaces=false,
  basicstyle=\footnotesize\ttfamily,
  keywordstyle=\bfseries\color{green!40!black},
  commentstyle=\itshape\color{purple!40!black},
  identifierstyle=\color{blue},
  stringstyle=\color{orange}
}

\begin{document}

\title{Tactics Arena}
\author{Cousin Brandon \and Chaudemanche Ewen \and Biardeau Tristan}
\date{8 Décembre} 

\maketitle
\chapter{Planning}

\textbf{Tristan ---> Refactoring grille + finition menu (retour arrière etc...) + gestion des couleurs}
\smallbreak
\textbf{Date limite: Dimanche 13 Décembre}

\bigbreak
\textbf{Ewen ---> Tests fonctions unités + Refactoring}
\smallbreak
\textbf{Date limite: Dimanche 13 Décembre}

\part{Affichage}

\chapter{Menu.c}
\vspace{-1cm}

\begin{lstlisting}[language=c]

void mainMenu(){
	/*
		Afficher menu principal
		1 - Nouvelle Partie
		2 - Charger une Partie
		3 - Quitter
	 */
}

void gameMenu(short noPlayer, movable, attackable){
	/*
		Afficher le menu de jeu
		1 - Unite pouvant se deplacer -> unitMenu();
		2 - Unite pouvant attaquer -> unitMenu();
		3 - Changer de direction -> unitMenu();
		4 - Passer tour
		5 - Abandonner la partie
	 */
}

void unitMenu(int choice, short noPlayer, movable, attackable){
	/*
		Afficher la liste des unites pouvant faire quelque chose
		Si choice-> 1 alors liste des unites pouvant se deplacer
		Si choice -> 2 alors liste des unites pouvant attaquer
		Si choice -> 3 alors liste de toutes les unites du joueur
		Appui sur une touche fait appel a playTurn();
	 */
}
\end{lstlisting}

\chapter{Grid.c}
\vspace{-1cm}

\begin{lstlisting}[language=c]

void gridDisp(){
	// Affiche la grille
}

void dispX(){
	// Affiche les coordonnees horizontales
}

void dispTile(unitName name){
	// Affiche une case
}

\end{lstlisting}

\part{Moteur de jeu}

\chapter{engine.c}
\vspace{-1cm}

\begin{lstlisting}[language=c]

bool gameInit(short * noPlayer){
	gridInit(); // Retourne un code d'erreur
	playerInit(); // Retourne un code d'erreur
	* noPlayer = rand(1,2); // Choisis le joueur qui commence en premier
	return true; // Si les fonctions ne retournent rien
}

void playersInit(){
	// Initialise les joueurs (listes + placement unites)
}

void playerAddUnit(short noPlayer, int * nbUnit){
	// Placement des unites par le joueur en cours
}

void gridInit(){
	// Initialise la grille
}

void selectUnit(vector * unitSelected, short noPlayer){
	// Selectionne l'unite -> coordonnees dans la matrice sous forme de vecteur
}

void playTurn(short noPlayer){
	unitAction movableUnits[NB_MAX_UNIT]; // Tableau des unites pouvant se deplacer + deplacement possible

	movable(movableUnits, noPlayer); // Retourne le tableau mis a jour + nombre d'unites pouvant se deplacer
	selectUnit(unitSelected, noPlayer); // Renvoie les coordonnees de l'unite selectionnee
	// Choisir l'action desiree 
	playMove(unitSelected, movableUnits);
	playAttack(unitSelected);
	playDirection(unitSelected);
}

bool endGame(short noPlayer){
	/* 	
		Test les conditions de victoire ou de match nul
	 	Affiche le vainqueur le cas echeant
	*/
}

void pathFind(vector unitSelected, vector fieldAction[]){
	// Cherche les chemins possibles pour l'unite correspondante
}

int movable(unitAction movableUnits[], short noPlayer){
	/* 	
		Retourne nombre d'unites deplacable + liste maj des unites pouvant se deplacer
	*/
}

void timeTurn(){
	// Gestion du temps
}

bool lookAround(vector currentUnit){
	/*
		Regarde autour de l'unite selectionnee si elle peut se deplacer sur une case si TP pas permise alors deplacement impossible
	 */
}

void playMove(vector unitSelected, unitAction path[]){
	/*
		Deplace l'unite selectionnee a l'endroit desire en prenant en compte les chemins possibles
	 */
}

void playAttack(vector unitSelected){
	/*
		Attaque avec l'aide de l'unite selectionnee une unite contenu dans la liste du joueur correspondant
	 */
}

void playDirection(vector unitSelected){
	//Change la direction de l'unite selectionnee
}

\end{lstlisting}


\part{Gestion des chaînes de caractères}

\chapter{manageString.c}
\vspace{-1cm}

\begin{lstlisting}[language=c]

void getCoordS(char coordString[], vector * coordUnit){
	// Recupere les coordonnees de l'unite a partir d'une chaine
}

char* get2Char(char name[]){
	// Recupere 2 caracteres pour l'affichage du nom de l'unite
}

char* getNameUnit(unitName name){
	// Recupere le nom de l'unite a partir de la liste enumeree
}

void printNameUnit(unitName name){
	// Affiche le nom de l'unite 
}

void isOutGrid(char * coordString){
	// Verifie que les coordonnees sont dans la grille
}

void correctCoord(char * coordString){
	// Verifie l'authenticite des coordonnees
}

void rightSide(char * coordString, short noPlayer){
	// Verifie que les coordonnees correspondent au bon camp
}

void clearBuffer(){
	// vide le tampon memoire
}

int read(char * string, short length){
	// Lecture securisee d'une chaine de caracteres
}

\end{lstlisting}

\part{Gestion du terminal}

\chapter{terminal.c}
\vspace{-1cm}

\begin{lstlisting}[language=c]

char * getColor(int color, char type[]){
	// Recupere le code correspondant a la couleur
}

void color(int color, char type[]){
	// Change la couleur de l'ecran ou de la police
}

void fontColor(int color){
	// Change la couleur de la police
}

void clearScreen(){
	// Efface l'ecran
}

\end{lstlisting}

\part{Gestion des listes}

\chapter{listes.c}
\vspace{-1cm}

\begin{lstlisting}[language=c]

void addUnit(short noPlayer, vector coordUnit){
	// Ajoute une unite dans la liste du joueur concernee
}

void printList(short noPlayer){
	// Affiche la liste des unites du joueur correspondant
}

void destroyUnit(short noPlayer, vector coordUnit){
	// Detruit une unite dans la liste du joueur correspondant
}

\end{lstlisting}

\part{Gestion des unités}

\chapter{unit.c}
\vspace{-1cm}

\begin{lstlisting}[language=c]

void unitInit(short noPlayer, vector coordUnit){
	// Initialise l'unite venant d'etre ajoutee
}

bool canGetPassed(unit * target){
	// Renvoie faux si l'unite ne peut etre traversee
}

bool canBlock(unit * target){
	// Renvoie faux si l'unite ne peut pas bloquer
}

bool canAttack(unit * target){
	// Renvoie faux si l'unite ne peut attaquer
}

bool canMove(unit * target){
	// Renvoie faux si l'unite ne peut bouger
}

void heal(vector source, short noPlayer){
	// Heal toutes les unites du joueur courant
}

int getSideAttacked(vector source, vector target){
	// Renvoie le cote cible par l'unite source
}

void attack(vector source, vector target){
	// Attaque l'unite cible avec l'unite source
	// Prendre en compte les capacites speciales
}

void copy(unit * destination, unit * source){
	// Copie l'integralite d'une structure vers une autre
}

void move(vector destination, vector source){
	// Deplace l'unite source vers le vecteur destination
}

void setDirection(vector source, short dir){
	// Definis la direction de l'unite
}


void addEffect(vector target, unitEffect effect){
	// Ajoute un effet sur l'unite
}

void AoE(vector target, int size, int dmg, bool own){
	// Attaque de zone
}

void line(vector source, int size, int dmg, int dir){
	// Attaque sur une ligne
}

\end{lstlisting}

\end{document}